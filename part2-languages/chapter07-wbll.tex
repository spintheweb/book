% Chapter 7: Webbase Layout Language (WBLL)

\chapter{Webbase Layout Language (WBLL)}
\label{chap:wbll}

\begin{quote}
\textit{"Simplicity is the ultimate sophistication."} \\
— Leonardo da Vinci
\end{quote}

The Webbase Layout Language (\textbf{WBLL}) defines how content is presented. Where \wbdl{} describes \textit{what} exists (site, areas, pages, contents), WBLL describes \textit{how} it looks and behaves at render time.

WBLL is a compact, token-based layout language parsed by the Web Spinner’s layout engine. It compiles into an efficient render function that merges data (records, session values, placeholders) into accessible, responsive HTML.

\section{Design Goals}
\label{sec:wbll-goals}

\begin{itemize}
  \item Declarative, compact, and readable by non-front-end specialists
  \item Safe by default: XSS-aware escaping and controlled injection points
  \item Data-driven: first-class support for records, fields, lists, and placeholders
  \item Composable: reusable fragments and wrappers
  \item Accessible: semantic outputs compatible with ARIA and keyboard navigation
\end{itemize}

\section{Layout Structure}
\label{sec:wbll-structure}

At a high level, a WBLL layout contains:

\begin{description}
  \item[Header/Footer Wrappers] Optional shells that frame the body
  \item[Body] The main fragment, typically a mix of text, tokens, and sections
  \item[Sections] Named containers (e.g., \texttt{header}, \texttt{body}, \texttt{footer}) to organize outputs
\end{description}

Typical WBLL is authored as a single string per language. The layout engine tokenizes and compiles it once, then reuses the compiled function.

\section{Placeholders and Data Binding}
\label{sec:wbll-placeholders}

WBLL works with the placeholders system used by \wbpl{}:

\begin{itemize}
  \item Session placeholders (user, roles, locale)
  \item Request placeholders (query/path params)
  \item Record placeholders (first row fields, list iteration variables)
\end{itemize}

Inline placeholders typically use the \verb|@@name| syntax. The engine escapes interpolated values by default; raw insertion is explicitly marked where needed.

\section{Core Token Categories}
\label{sec:wbll-tokens}

While concrete syntax may evolve, token categories are stable:

\subsection{Text and Headings}
\begin{itemize}
  \item Headings: \verb|#|, \verb|##|, \verb|###| → h1/h2/h3
  \item Inline emphasis: \verb|*italic*|, \verb|**bold**|
  \item Code spans and blocks for technical content
\end{itemize}

\subsection{Fields and Values}
\begin{itemize}
  \item Field: \verb|{{ fieldName }}| retrieves from the current record
  \item Session/Request: \verb|@@locale|, \verb|@@user|
  \item Formatting filters: e.g., \verb+{{ total | number:2 }}+
\end{itemize}

\subsection{Lists and Tables}
\begin{itemize}
  \item Iterate: \verb|{{#each rows}} ... {{/each}}|
  \item Empty state: \verb|{{#empty}} No results {{/empty}}|
  \item Table helpers: header/row/cell builders
\end{itemize}

\subsection{Links and Actions}
\begin{itemize}
  \item Internal links built from slugs and params
  \item Action buttons with method, payload, and confirmation
  \item Download links for binary responses
\end{itemize}

\subsection{Conditionals}
\begin{itemize}
  \item \verb|{{#if expr}} ... {{/if}}| with basic boolean logic
  \item \verb|{{#when role='admin'}} ... {{/when}}| role-scoped blocks
\end{itemize}

\section{Examples}
\label{sec:wbll-examples}

\subsection{Detail Card}
\begin{lstlisting}[language=HTML,caption={WBLL detail card (illustrative)}]
<section class="card">
  <h2>Order #{{ order_id }}</h2>
  <div class="grid">
    <div><strong>Date:</strong> {{ order_date | date }}</div>
    <div><strong>Customer:</strong> {{ customer_name }}</div>
    <div><strong>Total:</strong> {{ total | currency:'EUR' }}</div>
  </div>
  {{#when role='sales'}}
  <button data-action="close-order" data-id="{{ order_id }}">
    Close Order
  </button>
  {{/when}}
</section>
\end{lstlisting}

\subsection{Paginated Table}
\begin{lstlisting}[language=HTML,caption={WBLL list/table (illustrative)}]
<table class="table">
  <thead>
    <tr>
      <th>Id</th><th>Customer</th><th>Status</th><th>Total</th>
    </tr>
  </thead>
  <tbody>
    {{#each rows}}
    <tr>
      <td>{{ id }}</td>
      <td>{{ customer }}</td>
      <td>{{ status }}</td>
      <td>{{ total | number:2 }}</td>
    </tr>
    {{/each}}
    {{#empty}}
    <tr><td colspan="4">No orders found</td></tr>
    {{/empty}}
  </tbody>
</table>
<nav class="pager">
  {{#if has_prev}}
  <a href="?page={{ prev }}" aria-label="Previous">‹</a>
  {{/if}}
  <span>{{ page }} / {{ pages }}</span>
  {{#if has_next}}
  <a href="?page={{ next }}" aria-label="Next">›</a>
  {{/if}}
</nav>
\end{lstlisting}

\section{Security and Escaping}
\label{sec:wbll-security}

By default, interpolated values are HTML-escaped. Tokens that explicitly request raw HTML must be used sparingly and validated. Links/actions sanitize parameters to prevent injection.

\section{Internationalization}
\label{sec:wbll-i18n}

WBLL layouts are localized per language. Text nodes may be resolved through dictionaries; field formatting honors locale (dates, numbers, currency).

\section{Performance}
\label{sec:wbll-performance}

The engine compiles layouts into specialized functions and caches them until the template changes. Streaming responses are supported for large lists.

\section{Authoring Guidance}
\label{sec:wbll-guidance}

\begin{itemize}
  \item Keep structure semantic; use headings and lists appropriately
  \item Prefer built-in helpers over raw HTML for consistency and safety
  \item Provide empty states and loading/error placeholders
  \item Isolate reusable fragments; avoid duplication
\end{itemize}

This chapter establishes the presentation layer contracts. In the next chapters we detail modular composition (\cref{chap:webbaselets}) and the runtime engine (\cref{chap:web-spinner}).
