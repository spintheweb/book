% Chapter 2: The WBDL Language

\chapter{The WBDL Language}
\label{chap:wbdl}

\begin{quote}
\textit{"The best interface is no interface."} \\
— Golden Krishna
\end{quote}

\section{Introduction}

WBDL is formally defined using two standard schema languages: \textbf{XML Schema Definition (XSD)} and \textbf{JSON Schema}. This dual-schema approach ensures broad compatibility and facilitates validation and data exchange across different platforms and programming languages.

\section{The Book Analogy}

To better understand the structure of a web portal defined in WBDL, it's helpful to use the analogy of a book. The portal is organized hierarchically, much like a book is divided into chapters, pages, and paragraphs.

\begin{itemize}
\item \textbf{Areas (\texttt{STWArea})}: These are the main sections of the portal, analogous to the \textbf{chapters} of a book. An area groups related pages together.
\item \textbf{Pages (\texttt{STWPage})}: Contained within Areas, these are the individual \textbf{pages} of the book. Each page holds the actual content that users will see.
\item \textbf{Content (\texttt{STWContents})}: These are the building blocks of a page, similar to the \textbf{paragraphs} or other content elements (like images or tables) on a page.
\end{itemize}

\texttt{STWArea}, \texttt{STWPage}, and \texttt{STWContent} are all specialized types that inherit from the base \texttt{STWElement}, sharing its common properties while also having their own specific attributes and behaviors.

\section{Portal Organization and User Journeys}

The structure of a web portal built with WBDL is typically organized around the core functions of the business it represents. This creates a logical and intuitive navigation system for all users. Common top-level \textbf{Areas (\texttt{STWArea})} would include:

\begin{itemize}
\item Sales
\item Administration
\item Backoffice
\item Technical Office
\item Logistics
\item Products \& Services (often publicly viewable)
\end{itemize}

The full potential of the portal is revealed when we consider the specific journeys of different users, or \textbf{personas}. The portal uses a role-based system to present a completely different experience to each user, tailored to their needs and permissions.

\subsection{Example User Journeys}

\subsubsection{The Customer}
\begin{itemize}
\item Logs into the portal and is directed to a personalized ``Customer Dashboard'' page.
\item Can view their complete order history in a dedicated ``My Orders'' area.
\item Can track the real-time status of current orders (e.g., ``Processing,'' ``Shipped'').
\item Can initiate a video chat with their designated sales representative directly from the portal.
\item Can open a support ticket or schedule a consultation with the Technical Office.
\end{itemize}

\subsubsection{The Supplier}
\begin{itemize}
\item Logs in and sees a ``Supplier Dashboard.''
\item Can access a ``Kanban View'' page to see which materials or components require replenishment.
\item Can submit new quotations through an integrated form.
\item Can view the status of their invoices and payments.
\end{itemize}

\subsubsection{The Employee}
\begin{itemize}
\item Logs in and is presented with an ``Employee Self-Service'' area.
\item Can access a ``Welfare Management'' page to view and adjust their benefits.
\item Can view internal company news, submit vacation requests, and access HR documents.
\end{itemize}

\subsubsection{The CEO}
\begin{itemize}
\item Logs in to a high-level ``Executive Dashboard.''
\item Can view key performance indicators (KPIs) for the entire company, such as sales figures, production output, and financial health.
\item Can access detailed reports from various departments.
\end{itemize}

These user journeys demonstrate how WBDL's hierarchical structure and role-based visibility rules work together to create a highly functional and personalized web portal that serves as a central hub for the entire business ecosystem.

\section{Chapter Summary}

This chapter introduced the WBDL language, its hierarchical structure based on the book analogy, and how it enables role-based portal experiences for different user personas.

\textbf{Next}: In \chapref{chap:architecture}, we will examine the three-pillar architecture that makes this vision possible.
