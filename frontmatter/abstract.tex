% Abstract for the Spin the Web Book

\chapter*{Abstract}
\addcontentsline{toc}{chapter}{Abstract}
\textbf{Spin the Web} is an open source framework for building \textbf{Enterprise Web Portals} (``\gls{portal}''), with the intent of virtualizing the enterprise. In this book, a portal is a single access point that unifies the capabilities of websites and web applications: it aggregates and personalizes information, exposes interactive and transactional workflows, and serves as the enterprise brand's primary digital harbor. It addresses the persistent challenge of unifying heterogeneous enterprise systems (ERP, CRM, BPMS, and MRP systems) behind a single, role-aware digital channel, providing consistent abstractions over disparate backends. The framework is stewarded by the Spin the Web Foundation.

There are three core components:
\begin{enumerate}
\item \textbf{Webbase Description Language (WBDL)}: A declarative language for modeling portal structure, content, and behavior.
\item \textbf{Web Spinner}: A runtime that interprets WBDL (webbase) dynamically generating user experiences with real-time content delivery and role-based authorization.
\item \textbf{Spin the Web Studio}: An webbaselet for in-place editing of webbases; it also serves as a laboratory for exercising the Web Spinner.
\end{enumerate}

The framework advances the \textbf{Virtualized Enterprise paradigm}: a unified portal interface for customers, employees, suppliers, partners, and governance stakeholders, with the primary objective of digitally harboring the brand (\gls{ebranding}).

This book combines foundational concepts with practical guidance for developers, integrators, and technology leaders seeking to modernize digital channels.

\textbf{Keywords:} enterprise web portals; ebranding; webbase; webbaselets; framework; Spin the Web Foundation; STW.

\clearpage
