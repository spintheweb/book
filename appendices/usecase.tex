\chapter{Webbaselet Use Cases}
\label{app:usecases}

This appendix illustrates typical use cases for webbaselets:
small, self-contained web applications that used to live in
spreadsheets, ad-hoc scripts, or custom line-of-business tools.
The goal is not to replace existing systems, but to harmonize them:
software vendors expose stable APIs, and UI integrators compose
webbaselets that sit on top of those APIs.

Despite the name, webbaselets are not necessarily simple or small.
At one end of the spectrum they can model a single form or report;
at the other end they can embody a complete ERP, CRM, or vertical
line-of-business solution. The key distinction is not size, but the
fact that they are modular, API-driven, and composable.

\section{Team Task Board}

\subsection*{Before: Spreadsheet To-Do List}
Many teams maintain a shared spreadsheet to track tasks:
columns for assignee, status, due date, and comments. Over time,
these files proliferate (``Q1\_tasks\_final\_v7.xlsx''), access
control becomes informal, and simple workflows (e.g.\ notifications,
filtering by user) require manual effort.

\subsection*{After: Task Board Webbaselet}
A task board webbaselet connects to a tasks API (or a simple JSON
data source) and offers:
\begin{itemize}
  \item Column-based views (e.g.\ \emph{Backlog}, \emph{In Progress}, \emph{Done}).
  \item Per-user filters and saved views.
  \item Inline editing, without opening or emailing files.
  \item Optional integration with issue trackers or chat tools.
\end{itemize}
The underlying logic (task model, permissions, notifications)
remains in the core product; the webbaselet focuses on UI and
composition.

\section{Inventory and Stock Tracking}

\subsection*{Before: Warehouse Spreadsheet}
Small warehouses or teams often track inventory in a spreadsheet:
SKU, description, location, stock level, reorder threshold. There is
no single source of truth, concurrent edits conflict, and basic
queries (``show low stock by location'') require manual filtering.

\subsection*{After: Inventory Webbaselet}
An inventory webbaselet consumes an inventory API (or database
view) and provides:
\begin{itemize}
  \item Search and filtering by SKU, category, or location.
  \item Highlighting of low-stock and out-of-stock items.
  \item Simple forms for stock adjustments and movements.
  \item Read-only views for stakeholders who should not edit data.
\end{itemize}
The stock logic and validation stay in the backend; the webbaselet
offers a focused UI that can be embedded in portals or dashboards.

\section{Event Registration and Attendance}

\subsection*{Before: Sign-up Forms + CSV Exports}
Event organizers often rely on generic form builders that export
CSV files. They manually merge lists, send reminders, and track
attendance in a spreadsheet with ad-hoc columns.

\subsection*{After: Registration Webbaselet}
A registration webbaselet connects to an events/participants API
and enables:
\begin{itemize}
  \item Branded sign-up forms backed by a structured data model.
  \item Live views of registration counts per session or ticket type.
  \item Check-in views optimized for tablets or mobile devices.
  \item Export to downstream systems (CRM, email marketing) via APIs.
\end{itemize}
Again, business rules (capacity limits, ticket types, pricing) live
in the backend. The webbaselet tailors the UI for a specific event
or organizer.

\section{Lightweight CRM and Contact Lists}

\subsection*{Before: Contact Spreadsheet}
Sales or support teams maintain contact lists in spreadsheets: name,
company, email, notes, last-contact date. Different teams fork their
own copies, and no one has a consistent, up-to-date view of customer
interactions.

\subsection*{After: Contact Webbaselet}
A contact webbaselet integrates with a CRM or customer API and
offers:
\begin{itemize}
  \item Unified views of contacts with filters for segment, owner,
        or lifecycle stage.
  \item Inline note taking and tagging.
  \item Contextual links to external systems (support tickets,
        billing, product usage dashboards).
\end{itemize}
The CRM remains the system of record; the webbaselet is a customized
lens focused on one team’s workflow.

\section{Timesheets and Time Tracking}

\subsection*{Before: Weekly Timesheet Templates}
Consulting and development teams often email around weekly timesheet
templates as spreadsheets. Employees fill them in, send them back,
and someone consolidates the data manually for payroll or billing.

\subsection*{After: Timesheet Webbaselet}
A timesheet webbaselet talks to a time-tracking or billing API and
provides:
\begin{itemize}
  \item Simple weekly views per person, project, or client.
  \item Validation rules delegated to the backend (e.g.\ max hours,
        required fields).
  \item Role-specific views (employee, manager, finance).
  \item Export or synchronization to accounting systems.
\end{itemize}
The core product owns timesheet logic and compliance; the webbaselet
delivers a focused UI that can evolve independently.

\section{Guest Registration via QR Code}

\subsection*{Before: Paper Forms and Ad-hoc Apps}
Guest registration is often handled with paper sign-in sheets,
generic form builders, or improvised tablet apps. Data ends up
scattered across mailboxes and CSV exports; hosts lack a live
view of who has arrived, and compliance requirements (e.g.\ GDPR
consent, visitor badges, emergency lists) are handled informally.

\subsection*{After: Guest Registration Webbaselet}
A guest registration webbaselet focuses entirely on the arrival
experience and visitor lifecycle:
\begin{itemize}
  \item Pre-registration links sent by email, each with a unique QR code.
  \item A check-in view that scans the QR code and confirms arrival.
  \item Dynamic questions per visit type (e.g.\ NDA, health \& safety).
  \item Instant notifications to hosts and reception staff.
  \item Secure integration with identity systems and visitor logs via APIs.
\end{itemize}
The underlying policies (who may enter, retention times, badge
rules) remain in the core system; the webbaselet specializes the
UI around QR-code–based registration and check-in.

\section{Employee Timesheets}

\subsection*{Before: Spreadsheet-based Timesheets}
In many organizations, employees track working hours in shared
spreadsheets or emailed templates. Each week, files are copied,
renamed, and manually consolidated for payroll and billing. There
is no single source of truth, approval workflows are ad hoc, and
basic checks (e.g.\ missing entries, overtime rules) are performed
by hand.

\subsection*{After: Timesheet Webbaselet}
A timesheet webbaselet is dedicated to capturing, reviewing, and
exporting working hours:
\begin{itemize}
  \item Weekly and monthly views per employee, project, or client.
  \item Validation of entries (required fields, maximum hours, holidays)
        delegated to backend policies.
  \item Manager approval flows (pending, approved, rejected) surfaced
        as simple UI states.
  \item Role-specific perspectives for employees, managers, and finance.
  \item API-based integration with HR, payroll, and invoicing systems.
\end{itemize}
The core HR or ERP system remains the system of record; the webbaselet
provides a focused, evolvable UI for time capture and approval, instead
of relying on fragile spreadsheet processes.

\section{Human Resources and Welfare Management}

\subsection*{Before: Fragmented HR and Welfare Processes}
HR and welfare programs are often split across multiple tools:
spreadsheets for benefits eligibility, email threads for requests,
PDF forms for reimbursements, and separate portals for leave,
training, and well-being initiatives. Employees struggle to find
the right entry point, HR staff re-key data across systems, and
visibility on participation and costs is limited.

\subsection*{After: HR and Welfare Webbaselet}
An HR and welfare management webbaselet unifies access to policies,
requests, and status tracking:
\begin{itemize}
  \item A single entry point where employees can view benefits,
        submit requests (e.g.\ childcare, wellness, commuting),
        and track approvals.
  \item Dynamic workflows based on role, location, and employment
        type, driven by backend rules.
  \item Integration with HRIS, payroll, and benefits providers via
        APIs, avoiding manual data transfer.
  \item Aggregated dashboards for HR to monitor utilization,
        budget impact, and compliance indicators.
\end{itemize}
The core HR and payroll systems keep ownership of contracts,
eligibility, and calculations; the webbaselet orchestrates a
coherent, employee-centric UI for welfare management at scale.

\bigskip
These examples illustrate a recurring pattern: existing spreadsheets
and ad-hoc tools reveal stable data structures and workflows.
Webbaselets capture those patterns as reusable, composable UI pieces
that sit on top of well-defined APIs, harmonizing rather than
replacing the underlying systems.