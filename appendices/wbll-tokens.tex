% Appendix: WBLL Token Reference

\chapter{WBLL Token Reference}
\label{app:wbll-tokens}

This appendix provides a complete reference for WBLL tokens.

% Consistent spacing and styling for token boxes (transparent background)
\tcbset{
  tokenbox/.style={
    enhanced,
    colframe=black!10,
    boxrule=0.5pt,
    arc=2pt,
    before skip=2\baselineskip,
    after skip=2\baselineskip
  }
}

% Listings style for this appendix: no frame, wrap, NO background
\lstdefinestyle{wbll-appendix}{
  basicstyle=\ttfamily\small,
  upquote=true,
  breaklines=true,
  breakatwhitespace=true,
  columns=fullflexible,
  frame=none
}

% Inline WBLL syntax helper for use inside tables
\newcommand{\wballsyn}[1]{\lstinline[language=WBLL,basicstyle=\ttfamily\small]!#1!}

% General token syntax and notation
\noindent\textbf{General syntax:} \wballsyn{<token>('arg1[;arg2][;arg3]...')}

\noindent Brackets denote optional arguments, for example:
\begin{itemize}
  \item No arguments: \wballsyn{e}
  \item One argument: \wballsyn{e('format')}
  \item Two arguments: \wballsyn{e('format;name')}
  \item Three arguments: \wballsyn{e('format;name;value')}
\end{itemize}
If you need to skip an argument, use an empty segment: e.g., \wballsyn{e(';username')} omits \textit{format}.

\par\smallskip
\textit{Formatting note:} the \textit{format} argument applies only to numeric output in \texttt{e} and \texttt{f} (e.g., \texttt{€ \#,\#\#0.00}, \texttt{\#,\#\#0}, \texttt{0}). For non‑numeric values it is ignored; inputs default to \texttt{text}.
\par\smallskip
\textit{Cursor movement policy:} \textbf{editing tokens} (form controls such as \texttt{h}, \texttt{e}, \texttt{w}, \texttt{m}) advance the field cursor when the \emph{name} is not specified; \textbf{non-editing tokens} advance only when they implicitly consume the active field \emph{value}.

% Example recordset used in examples
\begin{tcolorbox}[tokenbox]
{\small\setstretch{1.05}%
\textbf{Example recordset} (fields in order): \texttt{price}, \texttt{username}, \texttt{city}.

\medskip
\begin{tabularx}{\linewidth}{@{}l l l@{}}
\textbf{price} & \textbf{username} & \textbf{city} \\
\hline
1234.5 & alice & Milan \\
99 & bob & Paris \\
0 & carol & Berlin \\
\end{tabularx}

\medskip
\textit{Assumptions for examples:} unless stated otherwise,
(1) examples evaluate the first row;
(2) the field cursor starts at the first field (\texttt{price});
(3) when currency output like “€ 1{,}234.50” appears, it is the formatted \texttt{price} of row 1 using the pattern \texttt{€ \#,\#\#0.00}.
}
\end{tcolorbox}

% /* */ — multiline comment (non-editing)
\begin{tcolorbox}[tokenbox]
{\small\setstretch{1.05}%
\begin{tabularx}{\linewidth}{@{}p{0.18\linewidth} X@{}}
\textbf{Token} & \wballsyn{/* comment text */} \\
\hline
\textbf{Description} & Multiline comment. The interpreter ignores everything between \texttt{/*} and \texttt{*/}. Can span multiple lines. Produces no output and does \textbf{not} move the field cursor. \\
\hline
\textbf{Example} & Code and output below. \\
\end{tabularx}}

\vspace{0.3em}
\begin{lstlisting}[style=wbll-appendix, language=WBLL]
lf /* comment
   spanning
   lines */
\rl('UserName')f
\end{lstlisting}
\textit{Renders:}
\begin{lstlisting}[style=wbll-appendix, language=HTML]
<label>price</label>1234.5<br><label>UserName</label>alice
\end{lstlisting}
\end{tcolorbox}

% // — line comment (non-editing)
\begin{tcolorbox}[tokenbox]
{\small\setstretch{1.05}%
\begin{tabularx}{\linewidth}{@{}p{0.18\linewidth} X@{}}
\textbf{Token} & \wballsyn{// comment text} \\
\hline
\textbf{Description} & Line comment. The interpreter ignores everything from \textbf{//} to the end of the line. Produces no output and does \textbf{not} move the field cursor. \\
\hline
\textbf{Example} & Code and output below. \\
\end{tabularx}}

\vspace{0.3em}
\begin{lstlisting}[style=wbll-appendix, language=WBLL]
t('Hello') // inline comment after code
\r
// full-line comment
t('World')
\end{lstlisting}
\textit{Renders:}
\begin{lstlisting}[style=wbll-appendix, language=HTML]
Hello<br>World
\end{lstlisting}
\end{tcolorbox}

% < — move field cursor backward
\begin{tcolorbox}[tokenbox]
{\small\setstretch{1.05}%
\begin{tabularx}{\linewidth}{@{}p{0.18\linewidth} X@{}}
\textbf{Token} & \wballsyn{<} \\
\hline
\textbf{Description} & Moves the field cursor back by one position. Produces no output. If the cursor is already at the first field, this operation is a no-op. \\
\hline
\textbf{Example} & Code and output below. \\
\end{tabularx}}

\vspace{0.3em}
\begin{lstlisting}[style=wbll-appendix, language=WBLL]
>>><f
\end{lstlisting}
\textit{Renders:}
\begin{lstlisting}[style=wbll-appendix, language=HTML]
Milan
\end{lstlisting}
\end{tcolorbox}

% > — advance field cursor
\begin{tcolorbox}[tokenbox]
{\small\setstretch{1.05}%
\begin{tabularx}{\linewidth}{@{}p{0.18\linewidth} X@{}}
\textbf{Token} & \wballsyn{>} \\
\hline
\textbf{Description} & Advances the field cursor by one position. Produces no output. If the cursor is already at the last field, this operation is a no-op. \\
\hline
\textbf{Example} & Code and output below. \\
\end{tabularx}}

\vspace{0.3em}
\begin{lstlisting}[style=wbll-appendix, language=WBLL]
>f
\end{lstlisting}
\textit{Renders:}
\begin{lstlisting}[style=wbll-appendix, language=HTML]
alice
\end{lstlisting}
\end{tcolorbox}

% \A — attributes for parent <tr> or <li> (non-editing)
\begin{tcolorbox}[tokenbox]
{\small\setstretch{1.05}%
\begin{tabularx}{\linewidth}{@{}p{0.18\linewidth} X@{}}
\textbf{Token} & \wballsyn{\\A('attr="value" ...')} \\
\hline
\textbf{Description} & Like \textbf{\textbackslash a}, but applies attributes to the \emph{parent} structural element: the nearest enclosing \texttt{<tr>} (in table contexts) or \texttt{<li>} (in list contexts). Non-editing; produces no output and does \textbf{not} move the field cursor. Has effect only when inside a table row or list item context. The attributes are processed by the WBPL interpreter. \\
\hline
\textbf{Example} & Code and output below. \\
\end{tabularx}}

\vspace{0.3em}
\begin{lstlisting}[style=wbll-appendix, language=WBLL]
// In a table row context (created by the table subtype)
\A('class="highlight"')e

// In a list item context (created by the list/menu subtype)
\A('data-role="user"')>e(';username')
\end{lstlisting}
\textit{Renders:}
\begin{lstlisting}[style=wbll-appendix, language=HTML]
<tr class="highlight">... <input type="text" name="price" value="1234.5"> ...</tr>
<li data-role="user">... <input type="text" name="username" value="alice"> ...</li>
\end{lstlisting}
\end{tcolorbox}

% \a — attributes for current element (non-editing)
\begin{tcolorbox}[tokenbox]
{\small\setstretch{1.05}%
\begin{tabularx}{\linewidth}{@{}p{0.18\linewidth} X@{}}
\textbf{Token} & \wballsyn{\\a('attr="value" ...')} \\
\hline
\textbf{Description} & Assigns HTML attributes to the most recently emitted element. Use immediately after a token that creates an element (e.g., \texttt{e}, \texttt{w}, \texttt{m}, \texttt{h}). Attributes are appended/merged; later attributes override earlier ones when duplicated. Non-editing; produces no output and does \textbf{not} move the field cursor. The attributes are processed by the WBPL interpreter. \\
\hline
\textbf{Example} & Code and output below. \\
\end{tabularx}}

\vspace{0.3em}
\begin{lstlisting}[style=wbll-appendix, language=WBLL]
e\a('required style="color:red"')
>e(';username')\a('maxlength="20"')
\end{lstlisting}
\textit{Renders:}
\begin{lstlisting}[style=wbll-appendix, language=HTML]
<input type="text" name="price" value="1234.5" required style="color:red"><input type="text" name="username" value="alice" maxlength="20">
\end{lstlisting}
\end{tcolorbox}

% \\r — line break (non-editing)
\begin{tcolorbox}[tokenbox]
{\small\setstretch{1.05}%
\begin{tabularx}{\linewidth}{@{}p{0.18\linewidth} X@{}}
\textbf{Token} & \wballsyn{\\r} \\
\hline
\textbf{Description} & Inserts a line break: \texttt{<br>}. Non-editing; produces no value output and does \textbf{not} advance the field cursor. \\
\hline
\textbf{Example} & Code and output below. \\
\end{tabularx}}

\vspace{0.3em}
\begin{lstlisting}[style=wbll-appendix, language=WBLL]
f\rf
\end{lstlisting}
\textit{Renders:}
\begin{lstlisting}[style=wbll-appendix, language=HTML]
1234.5<br>alice
\end{lstlisting}
\end{tcolorbox}

% e — text input
\begin{tcolorbox}[tokenbox]
{\small\setstretch{1.05}%
\begin{tabularx}{\linewidth}{@{}p{0.18\linewidth} X@{}}
\textbf{Token} & \wballsyn{e('format[;name][;value]')} \\
\hline
\textbf{Description} & Renders an input element. Arguments are \emph{positional but optional}. Defaults: \textit{format} = \texttt{text}, \textit{name} = active field name, \textit{value} = active field value. \textbf{Format options:} any HTML5 input type (\texttt{text}, \texttt{email}, \texttt{url}, \texttt{tel}, \texttt{number}, \texttt{date}, \texttt{datetime-local}, \texttt{time}, \texttt{month}, \texttt{week}, \texttt{color}, \texttt{range}, etc.); or a numeric/date-time formatting string (e.g., \texttt{\#,\#\#0.00} for decimals, \texttt{yyyy-MM-dd} for dates). Numeric patterns set \texttt{type="number"} with appropriate \texttt{step} attribute. \textbf{Cursor rule (editing):} advances when \textit{name} is not specified; if \textit{name} is provided, does not advance (even if \textit{value} is omitted). \textbf{Non-interactive mode:} renders as \textbf{f} (formatted field value). \\
\hline
\textbf{Example} & Code and output below. \\
\end{tabularx}}

\vspace{0.3em}
\begin{lstlisting}[style=wbll-appendix, language=WBLL]
e('email;userEmail') // HTML5 email input type
e('date;startDate') // HTML5 date input
e('#,##0.00;price;1234.5') // Numeric format pattern -> type="number" step="0.01"
e // Defaults to type="text", uses active field
\end{lstlisting}
\textit{Renders:}
\begin{lstlisting}[style=wbll-appendix, language=HTML]
<input type="email" name="userEmail" value="1234.5">
<input type="date" name="startDate" value="alice">
<input type="number" name="price" value="1234.5" step="0.01" inputmode="decimal">
<input type="text" name="price" value="1234.5">
\end{lstlisting}
\end{tcolorbox}

% f — formatted field value
\begin{tcolorbox}[tokenbox]
{\small\setstretch{1.05}%
\begin{tabularx}{\linewidth}{@{}p{0.18\linewidth} X@{}}
\textbf{Token} & \wballsyn{f('format')} \\
\hline
\textbf{Description} & Renders the active field value at the cursor, optionally applying a \textit{format}. The \textit{format} argument is positional but optional (default: identity). \textbf{Cursor rule (non-editing):} advances because it consumes the active field value. \textbf{Formatting:} applies to numeric and date-time values (e.g., \texttt{\#,\#\#0.00} for decimals, \texttt{yyyy-MM-dd} for dates). \\
\hline
\textbf{Example} & Code and output below. \\
\end{tabularx}}

\vspace{0.3em}
\begin{lstlisting}[style=wbll-appendix, language=WBLL]
f('#,##0.00')\r // Numeric formatting of active field
f('yyyy-MM-dd')\r // Date formatting
f // No format (identity)
\end{lstlisting}
\textit{Renders:}
\begin{lstlisting}[style=wbll-appendix, language=HTML]
1,234.50<br>2025-11-30<br>alice
\end{lstlisting}
\end{tcolorbox}

% h — hidden input
\begin{tcolorbox}[tokenbox]
{\small\setstretch{1.05}%
\begin{tabularx}{\linewidth}{@{}p{0.18\linewidth} X@{}}
\textbf{Token} & \wballsyn{h('name[;value]')} \\
\hline
\textbf{Description} & Inserts an HTML input element of type hidden. Arguments are \emph{positional but optional}. Defaults when omitted: \textit{name} = active field name, \textit{value} = active field value. \textbf{Cursor rule (editing):} the cursor advances when \textit{name} is not specified (e.g., just \textbf{h}); if \textit{name} is provided, the cursor does not advance. \\
\hline
\textbf{Example} & Code and output below. \\
\end{tabularx}}

\vspace{0.3em}
\begin{lstlisting}[style=wbll-appendix, language=WBLL]
h
h('kind;area')
h('id')
\end{lstlisting}
\textit{Renders:}
\begin{lstlisting}[style=wbll-appendix, language=HTML]
<input type="hidden" name="price" value="1234.5"><input type="hidden" name="kind" value="area"><input type="hidden" name="id" value="alice">
\end{lstlisting}
\end{tcolorbox}

% j — JavaScript code block (non-editing)
\begin{tcolorbox}[tokenbox]
{\small\setstretch{1.05}%
\begin{tabularx}{\linewidth}{@{}p{0.18\linewidth} X@{}}
\textbf{Token} & \wballsyn{j('code')} \\
\hline
\textbf{Description} & Inserts a JavaScript code block as \texttt{<script>code</script>}. If \textit{code} is omitted, uses the active field value. Placeholders (e.g., \texttt{@@placeholder}) within \textit{code} are automatically replaced with their values. Use \texttt{\textbackslash @} to escape the \texttt{@} symbol if literal text is needed. \textbf{Cursor rule (non-editing):} advances only when \textit{code} is empty (consuming active field value). \\
\hline
\textbf{Example} & Code and output below. \\
\end{tabularx}}

\vspace{0.3em}
\begin{lstlisting}[style=wbll-appendix, language=WBLL]
j('function foo() { alert("what?"); }')
j('console.log("Rows:", @@rows)')
j('// Email: user\@example.com')
j>
\end{lstlisting}
\textit{Renders (assuming @@rows = 3, active field value = "alert('Hello');"):}
\begin{lstlisting}[style=wbll-appendix, language=HTML]
<script>function foo() { alert("what?"); }</script>
<script>console.log("Rows:", 3)</script>
<script>// Email: user@example.com</script>
<script>alert('Hello');</script>
\end{lstlisting}
\end{tcolorbox}

% k — session variable (non-editing)
\begin{tcolorbox}[tokenbox]
{\small\setstretch{1.05}%
\begin{tabularx}{\linewidth}{@{}p{0.18\linewidth} X@{}}
\textbf{Token} & \wballsyn{k('name[;value]')} \\
\hline
\textbf{Description} & Sets a session variable. Arguments are \emph{positional but optional}. Defaults: \textit{name} = active field name, \textit{value} = active field value. Produces no output. \textbf{Cursor rule (non-editing):} advances only when it implicitly uses the active field \emph{value} (i.e., when \textit{value} is omitted). If \textit{value} is provided, the cursor does not advance. \\
\hline
\textbf{Example} & Code and output below. \\
\end{tabularx}}

\vspace{0.3em}
\begin{lstlisting}[style=wbll-appendix, language=WBLL]
k('role;admin')
k
\end{lstlisting}
\textit{Renders:}
\begin{lstlisting}[style=wbll-appendix, language=HTML]
\end{lstlisting}
\end{tcolorbox}

% l — label sibling
\begin{tcolorbox}[tokenbox]
{\small\setstretch{1.05}%
\begin{tabularx}{\linewidth}{@{}p{0.18\linewidth} X@{}}
\textbf{Token} & \wballsyn{l('id[;text]')} \\
\hline
\textbf{Description} & Renders a label element \emph{as a sibling} (not wrapping) to the next input element: \texttt{<label for="<id>"><text></label>}. The \texttt{for} attribute references the next input's \texttt{id}. Arguments: \textit{id} (optional, defaults to auto-generated ID), \textit{text} (optional, defaults to active field name or value). \textbf{Cursor rule (non-editing):} does not advance. \textbf{Note:} Use uppercase \textbf{L} to wrap the next input inside the label. \\
\hline
\textbf{Example} & Code and output below. \\
\end{tabularx}}

\vspace{0.3em}
\begin{lstlisting}[style=wbll-appendix, language=WBLL]
l('uname;Username:')e('username')
\end{lstlisting}
\textit{Renders:}
\begin{lstlisting}[style=wbll-appendix, language=HTML]
<label for="uname">Username:</label><input type="text" name="username" id="uname" value="1234.5">
\end{lstlisting}
\end{tcolorbox}

% L — label wrapper
\begin{tcolorbox}[tokenbox]
{\small\setstretch{1.05}%
\begin{tabularx}{\linewidth}{@{}p{0.18\linewidth} X@{}}
\textbf{Token} & \wballsyn{L('text')} \\
\hline
\textbf{Description} & Renders a label element that \emph{wraps} the next input element: \texttt{<label><text> <input...></label>}. The \textit{text} argument is optional (default = active field value or empty). Unlike lowercase \textbf{l}, this creates a label without \texttt{for} attribute and contains the input. \textbf{Cursor rule (non-editing):} does not advance. \\
\hline
\textbf{Example} & Code and output below. \\
\end{tabularx}}

\vspace{0.3em}
\begin{lstlisting}[style=wbll-appendix, language=WBLL]
L('Username:')e('username')
\end{lstlisting}
\textit{Renders:}
\begin{lstlisting}[style=wbll-appendix, language=HTML]
<label>Username: <input type="text" name="username" value="1234.5"></label>
\end{lstlisting}
\end{tcolorbox}

% m — multiline text area / editor
\begin{tcolorbox}[tokenbox]
{\small\setstretch{1.05}%
\begin{tabularx}{\linewidth}{@{}p{0.18\linewidth} X@{}}
\textbf{Token} & \wballsyn{m('format[;name][;value]')} \\
\hline
\textbf{Description} & Renders a multiline input. Arguments are \emph{positional but optional}. \textbf{Format modes:} empty or \texttt{textarea} = plain \texttt{<textarea>}; \texttt{wysiwyg} = Summernote WYSIWYG editor; language code (\texttt{javascript}, \texttt{html}, \texttt{css}, \texttt{bash}, etc.) = Ace code editor with syntax highlighting. Defaults: \textit{format} = textarea, \textit{name} = active field name, \textit{value} = active field value. \textbf{Cursor rule (editing):} advances when \textit{name} is not specified; if \textit{name} is provided, does not advance. \textbf{Non-interactive mode:} renders as \textbf{f} (formatted field value). \\
\hline
\textbf{Example} & Code and output below. \\
\end{tabularx}}

\vspace{0.3em}
\begin{lstlisting}[style=wbll-appendix, language=WBLL]
m(';comments;Hello') // Plain textarea
m('wysiwyg;description') // Summernote editor
m('javascript;code;console.log("test")') // Ace editor with JavaScript syntax
\end{lstlisting}
\textit{Renders:}
\begin{lstlisting}[style=wbll-appendix, language=HTML]
<textarea name="comments">Hello</textarea>
<textarea id="..." name="description">alice</textarea><script>...</script>
<div id="..." class="stwCodeeditor">...</div><textarea id="..." name="code" style="display:none">console.log("test")</textarea><script>...</script>
\end{lstlisting}
\end{tcolorbox}

% \s — content attributes
\begin{tcolorbox}[tokenbox]
{\small\setstretch{1.05}%
\begin{tabularx}{\linewidth}{@{}p{0.18\linewidth} X@{}}
\textbf{Token} & \wballsyn{\\s('attr="value" ...')} \\
\hline
\textbf{Description} & Sets content-level attributes: \texttt{caption}, \texttt{header}, \texttt{footer}, \texttt{key}, \texttt{visible}, \texttt{enabled}, \texttt{disabled}, \texttt{nodata}. Non-editing; does not move the field cursor. Multiple calls accumulate; later values override earlier ones for the same key. \\
\hline
\textbf{Example} & Code and output below. \\
\end{tabularx}}

\vspace{0.3em}
\begin{lstlisting}[style=wbll-appendix, language=WBLL]
\s('caption="Users" header="Found: @@rows" key="fId"') l f l f l f
\end{lstlisting}
\textit{Renders (table subtype):}
\begin{lstlisting}[style=wbll-appendix, language=HTML]
<table data-key="fId">
  <caption>Users</caption>
  <thead>Found: 3</thead>
  <tbody>
    <tr>
      <th>price</th><td>1234.5</td>
      <th>username</th><td>alice</td>
      <th>city</th><td>Milan</td>
    </tr>
    ...
  </tbody>
</table>
\end{lstlisting}
\end{tcolorbox}

% t — text
\begin{tcolorbox}[tokenbox]
{\small\setstretch{1.05}%
\begin{tabularx}{\linewidth}{@{}p{0.18\linewidth} X@{}}
\textbf{Token} & \wballsyn{t('text')} \\
\hline
\textbf{Description} & Inserts the given text string into the response flow at the current position. Non-editing; does not move the field cursor. \\
\hline
\textbf{Example} & Code and output below. \\
\end{tabularx}}

% Keep verbatim listings outside tabularx to prevent runaway arguments
\vspace{0.3em}
\begin{lstlisting}[style=wbll-appendix, language=WBLL]
t('Hello World!')
\end{lstlisting}
\textit{Renders:}
\begin{lstlisting}[style=wbll-appendix, language=HTML]
Hello World!
\end{lstlisting}
\end{tcolorbox}

% w — password input
\begin{tcolorbox}[tokenbox]
{\small\setstretch{1.05}%
\begin{tabularx}{\linewidth}{@{}p{0.18\linewidth} X@{}}
\textbf{Token} & \wballsyn{w('name[;value]')} \\
\hline
\textbf{Description} & Renders a password input element: \texttt{<input type="password" name="<name>" value="<value>">}. Arguments are \emph{positional but optional}. Defaults: \textit{name} = active field name, \textit{value} = active field value. \textbf{Cursor rule (editing):} advances when \textit{name} is not specified; if \textit{name} is provided, does not advance. \textit{Note:} some browsers ignore preset password values. \\
\hline
\textbf{Example} & Code and output below. \\
\end{tabularx}}

\vspace{0.3em}
\begin{lstlisting}[style=wbll-appendix, language=WBLL]
w('pwd;secret')\rw\rw(';password')
\end{lstlisting}
\textit{Renders:}
\begin{lstlisting}[style=wbll-appendix, language=HTML]
<input type="password" name="pwd" value="secret"><br><input type="password" name="price" value="1234.5"><br><input type="password" name="username" value="password">
\end{lstlisting}
\end{tcolorbox}

% a — anchor/link
\begin{tcolorbox}[tokenbox]
{\small\setstretch{1.05}%
\begin{tabularx}{\linewidth}{@{}p{0.18\linewidth} X@{}}
\textbf{Token} & \wballsyn{a('url')p('name[;value]')t('text')} \\
\hline
\textbf{Description} & Renders an anchor element: \texttt{<a href="url?params">text</a>}. The \textit{url} argument is optional (default = active field value). Use \textbf{p()} tokens to append query parameters; use \textbf{t()} or another display token for link text. \textbf{Cursor rule (non-editing):} advances only if \textit{url} is empty (consuming active field value). \textbf{Note:} Use uppercase \textbf{A} to open link in new tab. \\
\hline
\textbf{Example} & Code and output below. \\
\end{tabularx}}

\vspace{0.3em}
\begin{lstlisting}[style=wbll-appendix, language=WBLL]
a('/users')p('id;@@userId')t('View User')
a>t('Link from field')
\end{lstlisting}
\textit{Renders:}
\begin{lstlisting}[style=wbll-appendix, language=HTML]
<a href="/users?id=123">View User</a>
<a href="1234.5">Link from field</a>
\end{lstlisting}
\end{tcolorbox}

% A — anchor/link (new tab)
\begin{tcolorbox}[tokenbox]
{\small\setstretch{1.05}%
\begin{tabularx}{\linewidth}{@{}p{0.18\linewidth} X@{}}
\textbf{Token} & \wballsyn{A('url')p('name[;value]')t('text')} \\
\hline
\textbf{Description} & Renders an anchor element with \texttt{target="\_blank"}: \texttt{<a href="url?params" target="\_blank">text</a>}. Same as lowercase \textbf{a} but opens in a new tab/window. \textbf{Cursor rule (non-editing):} advances only if \textit{url} is empty. \\
\hline
\textbf{Example} & Code and output below. \\
\end{tabularx}}

\vspace{0.3em}
\begin{lstlisting}[style=wbll-appendix, language=WBLL]
A('https://example.com')t('External Link')
\end{lstlisting}
\textit{Renders:}
\begin{lstlisting}[style=wbll-appendix, language=HTML]
<a href="https://example.com" target="_blank">External Link</a>
\end{lstlisting}
\end{tcolorbox}

% p — parameter (for links and buttons)
\begin{tcolorbox}[tokenbox]
{\small\setstretch{1.05}%
\begin{tabularx}{\linewidth}{@{}p{0.18\linewidth} X@{}}
\textbf{Token} & \wballsyn{p('name[;value]')} \\
\hline
\textbf{Description} & Appends a query parameter to the preceding \textbf{a}, \textbf{A}, \textbf{b}, or \textbf{o} token. Arguments: \textit{name} (required), \textit{value} (optional, defaults to active field value or placeholder if \texttt{@@name}). Does not produce output directly; modifies the URL/href of the parent element. \textbf{Cursor rule (non-editing):} advances only when \textit{value} is empty (consuming active field value). \\
\hline
\textbf{Example} & See examples for \textbf{a}, \textbf{b}, \textbf{o} tokens. \\
\end{tabularx}}
\end{tcolorbox}

% b — button
\begin{tcolorbox}[tokenbox]
{\small\setstretch{1.05}%
\begin{tabularx}{\linewidth}{@{}p{0.18\linewidth} X@{}}
\textbf{Token} & \wballsyn{b('url;namespace;action')p('name[;value]')t('text')} \\
\hline
\textbf{Description} & Renders a button or submit input. Arguments: \textit{url} (optional redirect), \textit{namespace} (typically \texttt{stw}), \textit{action} (e.g., \texttt{insert}, \texttt{update}, \texttt{delete}, \texttt{search}, \texttt{filter}, \texttt{submit}, \texttt{logon}, \texttt{logoff}, \texttt{pwdreset}). Use \textbf{p()} for parameters, \textbf{t()} for button text. \textbf{Cursor rule (non-editing):} does not advance. \\
\hline
\textbf{Example} & Code and output below. \\
\end{tabularx}}

\vspace{0.3em}
\begin{lstlisting}[style=wbll-appendix, language=WBLL]
b(';stw;insert')t('Add Record')
b(';stw;submit')t('Save')
\end{lstlisting}
\textit{Renders:}
\begin{lstlisting}[style=wbll-appendix, language=HTML]
<button type="submit" name="stwAction" value="stwinsert">Add Record</button>
<button type="submit" name="stwAction" value="stwsubmit">Save</button>
\end{lstlisting}
\end{tcolorbox}

% c — checkbox
\begin{tcolorbox}[tokenbox]
{\small\setstretch{1.05}%
\begin{tabularx}{\linewidth}{@{}p{0.18\linewidth} X@{}}
\textbf{Token} & \wballsyn{c('mode;name;value;options...')} \\
\hline
\textbf{Description} & Renders checkbox input(s). \textbf{Option modes:} \texttt{mode=0} = datasource query (\texttt{dsn;SELECT ...}); \texttt{mode=1} = simple list (each item is both value and label); \texttt{mode=2} = key-value pairs (value;label;value;label...). For datasource mode, if query returns 1 field, acts as mode=1; if 2 fields (id, value), acts as mode=2. Arguments: \textit{mode}, \textit{name} (optional, defaults to active field), \textit{value} (optional, defaults to active field value), then options. \textbf{Cursor rule (editing):} advances when \textit{name} is not specified. \textbf{Non-interactive mode:} renders as disabled checkboxes. \\
\hline
\textbf{Example} & Code and output below. \\
\end{tabularx}}

\vspace{0.3em}
\begin{lstlisting}[style=wbll-appendix, language=WBLL]
c('0;tags;;mysqldb;SELECT id,name FROM tags') // Datasource (2 fields -> key-value)
c('1;agree;;yes;no;maybe') // Simple list
c('2;choice;;1;Option A;2;Option B') // Key-value pairs
\end{lstlisting}
\textit{Renders:}
\begin{lstlisting}[style=wbll-appendix, language=HTML]
<fieldset class="stwGroup"><label><input type="checkbox" name="tags" value="1">Tag One</label>...</fieldset>
<fieldset class="stwGroup"><label><input type="checkbox" name="agree" value="yes">yes</label>...</fieldset>
<fieldset class="stwGroup"><label><input type="checkbox" name="choice" value="1">Option A</label>...</fieldset>
\end{lstlisting}
\end{tcolorbox}

% r — radio button
\begin{tcolorbox}[tokenbox]
{\small\setstretch{1.05}%
\begin{tabularx}{\linewidth}{@{}p{0.18\linewidth} X@{}}
\textbf{Token} & \wballsyn{r('mode;name;value;options...')} \\
\hline
\textbf{Description} & Renders radio button input(s). \textbf{Option modes:} \texttt{mode=0} = datasource query (\texttt{dsn;SELECT ...}); \texttt{mode=1} = simple list; \texttt{mode=2} = key-value pairs. For datasource mode, if query returns 1 field, acts as mode=1; if 2 fields, acts as mode=2. Arguments: \textit{mode}, \textit{name} (optional), \textit{value} (optional), then options. \textbf{Cursor rule (editing):} advances when \textit{name} is not specified. \textbf{Non-interactive mode:} renders as disabled radio buttons. \\
\hline
\textbf{Example} & Code and output below. \\
\end{tabularx}}

\vspace{0.3em}
\begin{lstlisting}[style=wbll-appendix, language=WBLL]
r('1;gender;;male;female;other')
\end{lstlisting}
\textit{Renders:}
\begin{lstlisting}[style=wbll-appendix, language=HTML]
<fieldset class="stwGroup"><label><input type="radio" name="gender" value="male">male</label>...</fieldset>
\end{lstlisting}
\end{tcolorbox}

% d/D — dropdown select
\begin{tcolorbox}[tokenbox]
{\small\setstretch{1.05}%
\begin{tabularx}{\linewidth}{@{}p{0.18\linewidth} X@{}}
\textbf{Token} & \wballsyn{d('name[;value];mode;options...')} or \wballsyn{D(...)} \\
\hline
\textbf{Description} & Renders a dropdown select. \textbf{Lowercase d} allows empty selection (includes \texttt{<option></option>}); \textbf{uppercase D} requires selection (no empty option). \textbf{Option modes:} \texttt{mode=0} = datasource query (\texttt{dsn;SELECT ...}); \texttt{mode=1} = simple list; \texttt{mode=2} = key-value pairs. For datasource mode, if query returns 1 field, acts as mode=1; if 2 fields, acts as mode=2. Arguments: \textit{name} (optional), \textit{value} (optional), \textit{mode}, then options. \textbf{Cursor rule (editing):} advances when \textit{name} is not specified. \textbf{Non-interactive mode:} renders as \textbf{n} (mapped value display). \\
\hline
\textbf{Example} & Code and output below. \\
\end{tabularx}}

\vspace{0.3em}
\begin{lstlisting}[style=wbll-appendix, language=WBLL]
d('country;;0;mysqldb;SELECT id,name FROM countries') // Datasource
d('city;;1;Milan;Paris;Berlin') // Simple list
D('status;;2;1;Active;2;Inactive') // Key-value pairs
\end{lstlisting}
\textit{Renders:}
\begin{lstlisting}[style=wbll-appendix, language=HTML]
<select name="country"><option></option><option value="1">Italy</option>...</select>
<select name="city"><option></option><option value="Milan">Milan</option>...</select>
<select name="status"><option value="1">Active</option><option value="2">Inactive</option></select>
\end{lstlisting}
\end{tcolorbox}

% s/S — multi-select
\begin{tcolorbox}[tokenbox]
{\small\setstretch{1.05}%
\begin{tabularx}{\linewidth}{@{}p{0.18\linewidth} X@{}}
\textbf{Token} & \wballsyn{s('name[;value];mode;options...')} or \wballsyn{S(...)} \\
\hline
\textbf{Description} & Renders a multi-select dropdown with \texttt{multiple} attribute. \textbf{Lowercase s} allows empty; \textbf{uppercase S} requires selection. \textbf{Option modes:} \texttt{mode=0} = datasource query (\texttt{dsn;SELECT ...}); \texttt{mode=1} = simple list; \texttt{mode=2} = key-value pairs. For datasource mode, if query returns 1 field, acts as mode=1; if 2 fields, acts as mode=2. \textbf{Cursor rule (editing):} advances when \textit{name} is not specified. \textbf{Non-interactive mode:} renders comma-separated mapped values (similar to \textbf{n} but handles multiple selections). \\
\hline
\textbf{Example} & Code and output below. \\
\end{tabularx}}

\vspace{0.3em}
\begin{lstlisting}[style=wbll-appendix, language=WBLL]
s('tags;;1;tag1;tag2;tag3')
\end{lstlisting}
\textit{Renders:}
\begin{lstlisting}[style=wbll-appendix, language=HTML]
<select name="tags" multiple><option></option><option value="tag1">tag1</option>...</select>
\end{lstlisting}
\end{tcolorbox}

% u — file upload
\begin{tcolorbox}[tokenbox]
{\small\setstretch{1.05}%
\begin{tabularx}{\linewidth}{@{}p{0.18\linewidth} X@{}}
\textbf{Token} & \wballsyn{u} \\
\hline
\textbf{Description} & Renders a file upload input: \texttt{<input type="file">}. Uses active field name. \textbf{Cursor rule (non-editing):} does not advance. \\
\hline
\textbf{Example} & Code and output below. \\
\end{tabularx}}

\vspace{0.3em}
\begin{lstlisting}[style=wbll-appendix, language=WBLL]
u\a('accept="image/*"')
\end{lstlisting}
\textit{Renders:}
\begin{lstlisting}[style=wbll-appendix, language=HTML]
<input type="file" accept="image/*">
\end{lstlisting}
\end{tcolorbox}

% i — image
\begin{tcolorbox}[tokenbox]
{\small\setstretch{1.05}%
\begin{tabularx}{\linewidth}{@{}p{0.18\linewidth} X@{}}
\textbf{Token} & \wballsyn{i('mode;options...')} \\
\hline
\textbf{Description} & Renders an image element. \textbf{Modes:} empty or \texttt{0} = direct URL from \textit{mode} or active field value; \texttt{1} = indexed by value (field value selects from option list); \texttt{2} = key-value pairs (field value matches key, renders corresponding URL). \textbf{Cursor rule (non-editing):} advances. \\
\hline
\textbf{Example} & Code and output below. \\
\end{tabularx}}

\vspace{0.3em}
\begin{lstlisting}[style=wbll-appendix, language=WBLL]
i('/img/logo.png')
i('1;img1.png;img2.png;img3.png')
i('2;1;img-one.png;2;img-two.png')
\end{lstlisting}
\textit{Renders (assuming field value = 2):}
\begin{lstlisting}[style=wbll-appendix, language=HTML]
<img src="/img/logo.png">
<img src="img2.png">
<img src="img-two.png">
\end{lstlisting}
\end{tcolorbox}

% n — mapped value display
\begin{tcolorbox}[tokenbox]
{\small\setstretch{1.05}%
\begin{tabularx}{\linewidth}{@{}p{0.18\linewidth} X@{}}
\textbf{Token} & \wballsyn{n('mode;options...')} \\
\hline
\textbf{Description} & Displays a mapped value (non-interactive version of dropdown). \textbf{Modes:} \texttt{1} = indexed by value; \texttt{2} = key-value pairs. \textbf{Cursor rule (non-editing):} advances. \\
\hline
\textbf{Example} & Code and output below. \\
\end{tabularx}}

\vspace{0.3em}
\begin{lstlisting}[style=wbll-appendix, language=WBLL]
n('1;Low;Medium;High')
n('2;1;Active;2;Inactive;3;Pending')
\end{lstlisting}
\textit{Renders (field value = 2):}
\begin{lstlisting}[style=wbll-appendix, language=HTML]
Medium
Inactive
\end{lstlisting}
\end{tcolorbox}

% o — article/container
\begin{tcolorbox}[tokenbox]
{\small\setstretch{1.05}%
\begin{tabularx}{\linewidth}{@{}p{0.18\linewidth} X@{}}
\textbf{Token} & \wballsyn{o('url')p('name[;value]')} \\
\hline
\textbf{Description} & Renders an article container element with auto-generated ID and optional href built from \textit{url} and parameters. \textbf{Cursor rule (non-editing):} does not advance. \\
\hline
\textbf{Example} & Code and output below. \\
\end{tabularx}}

\vspace{0.3em}
\begin{lstlisting}[style=wbll-appendix, language=WBLL]
o('/item')p('id')
\end{lstlisting}
\textit{Renders:}
\begin{lstlisting}[style=wbll-appendix, language=HTML]
<article id="..." href="/item?id=1234.5"></article>
\end{lstlisting}
\end{tcolorbox}

% v — eval expression
\begin{tcolorbox}[tokenbox]
{\small\setstretch{1.05}%
\begin{tabularx}{\linewidth}{@{}p{0.18\linewidth} X@{}}
\textbf{Token} & \wballsyn{v('expression')} \\
\hline
\textbf{Description} & Evaluates a JavaScript expression at compile time and inlines the result. \textbf{Warning:} executes at layout compilation time, not render time. Use with caution. \textbf{Cursor rule (non-editing):} does not advance. \\
\hline
\textbf{Example} & Code and output below. \\
\end{tabularx}}

\vspace{0.3em}
\begin{lstlisting}[style=wbll-appendix, language=WBLL]
v('new Date().getFullYear()')
\end{lstlisting}
\textit{Renders:}
\begin{lstlisting}[style=wbll-appendix, language=HTML]
2025
\end{lstlisting}
\end{tcolorbox}

% \n — line break
\begin{tcolorbox}[tokenbox]
{\small\setstretch{1.05}%
\begin{tabularx}{\linewidth}{@{}p{0.18\linewidth} X@{}}
\textbf{Token} & \wballsyn{\\n} \\
\hline
\textbf{Description} & Inserts a line break: \texttt{<br>}. Same as \textbf{\textbackslash r}. Non-editing; does not advance the field cursor. \\
\hline
\textbf{Example} & Code and output below. \\
\end{tabularx}}

\vspace{0.3em}
\begin{lstlisting}[style=wbll-appendix, language=WBLL]
t('Line 1')\nt('Line 2')
\end{lstlisting}
\textit{Renders:}
\begin{lstlisting}[style=wbll-appendix, language=HTML]
Line 1<br>Line 2
\end{lstlisting}
\end{tcolorbox}

% \t — empty label spacer
\begin{tcolorbox}[tokenbox]
{\small\setstretch{1.05}%
\begin{tabularx}{\linewidth}{@{}p{0.18\linewidth} X@{}}
\textbf{Token} & \wballsyn{\\t} \\
\hline
\textbf{Description} & Inserts an empty label element: \texttt{<label></label>}. Useful for layout spacing in flex forms. Non-editing; does not advance the field cursor. \\
\hline
\textbf{Example} & Code and output below. \\
\end{tabularx}}

\vspace{0.3em}
\begin{lstlisting}[style=wbll-appendix, language=WBLL]
\tL('Name:')e
\end{lstlisting}
\textit{Renders:}
\begin{lstlisting}[style=wbll-appendix, language=HTML]
<label></label><label>Name: <input type="text" name="price" value="1234.5"></label>
\end{lstlisting}
\end{tcolorbox}

% x — horizontal bar (for plots)
\begin{tcolorbox}[tokenbox]
{\small\setstretch{1.05}%
\begin{tabularx}{\linewidth}{@{}p{0.18\linewidth} X@{}}
\textbf{Token} & \wballsyn{x('width')} \\
\hline
\textbf{Description} & Renders a horizontal bar element for plots and visualizations: \texttt{<span style="display:inline-block; width:<width>px; height:1rem;"></span>}. The \textit{width} argument (in pixels) is optional; default = active field value. \textbf{Cursor rule (non-editing):} advances when \textit{width} is not specified (consuming active field value). \\
\hline
\textbf{Example} & Code and output below. \\
\end{tabularx}}

\vspace{0.3em}
\begin{lstlisting}[style=wbll-appendix, language=WBLL]
x('100')
x
\end{lstlisting}
\textit{Renders (assuming active field value = 250):}
\begin{lstlisting}[style=wbll-appendix, language=HTML]
<span style="display:inline-block; width:100px; height:1rem;"></span>
<span style="display:inline-block; width:250px; height:1rem;"></span>
\end{lstlisting}
\end{tcolorbox}

% y — vertical bar (for plots)
\begin{tcolorbox}[tokenbox]
{\small\setstretch{1.05}%
\begin{tabularx}{\linewidth}{@{}p{0.18\linewidth} X@{}}
\textbf{Token} & \wballsyn{y('height')} \\
\hline
\textbf{Description} & Renders a vertical bar element for plots and visualizations: \texttt{<span style="display:inline-block; width:1rem; height:<height>px;"></span>}. The \textit{height} argument (in pixels) is optional; default = active field value. \textbf{Cursor rule (non-editing):} advances when \textit{height} is not specified (consuming active field value). \\
\hline
\textbf{Example} & Code and output below. \\
\end{tabularx}}

\vspace{0.3em}
\begin{lstlisting}[style=wbll-appendix, language=WBLL]
y('150')
y
\end{lstlisting}
\textit{Renders (assuming active field value = 200):}
\begin{lstlisting}[style=wbll-appendix, language=HTML]
<span style="display:inline-block; width:1rem; height:150px;"></span>
<span style="display:inline-block; width:1rem; height:200px;"></span>
\end{lstlisting}
\end{tcolorbox}

% z — reserved/no-op
\begin{tcolorbox}[tokenbox]
{\small\setstretch{1.05}%
\begin{tabularx}{\linewidth}{@{}p{0.18\linewidth} X@{}}
\textbf{Token} & \wballsyn{z} \\
\hline
\textbf{Description} & Reserved token; currently no-op (produces no output). May be used for future extensions. Non-editing; does not advance the field cursor. \\
\hline
\textbf{Example} & No output. \\
\end{tabularx}}
\end{tcolorbox}

