% Part II: The Webbase Description Language (WBDL)

\chapter*{Introduction to Part II}
\addcontentsline{toc}{chapter}{Introduction to Part II}

The Webbase Description Language (\wbdl{}) forms the declarative foundation of the Spin the Web Project. This part explores the complete \wbdl{} specification, from its core element types and hierarchical structure to advanced concepts like webbaselets and modular portal development.

\section*{Part Structure}

This part consists of three comprehensive chapters:

\begin{description}
\item[\textbf{Chapter 4: The WBDL Language}] (\cref{chap:wbdl-language}) -- Introduces the complete \wbdl{} specification, including the Spin the Web Studio, the \texttt{STWElement} base type and all specialized element types (\texttt{STWSite}, \texttt{STWArea}, \texttt{STWPage}, \texttt{STWContent}). This chapter provides the detailed XSD definitions and property descriptions that form the language foundation.

\item[\textbf{Chapter 5: The Webbase Placeholders Language (WBPL)}] (\cref{chap:wbpl}) -- Explores the dynamic query processing capabilities of \wbpl{}, including placeholder syntax, substitution mechanisms, conditional blocks, and security features that enable data-driven content generation.

\item[\textbf{Chapter 6: Webbase and Webbaselets}] (\cref{chap:webbase-webbaselets}) -- Covers the modular architecture concepts that enable scalable portal development, including webbase structure, webbaselet integration patterns, development workflows, and governance approaches.
\end{description}

\section*{Learning Objectives}

By the end of this part, readers will have mastered:

\begin{itemize}
    \item The complete \wbdl{} language specification and schema definitions
    \item How to design and structure portal hierarchies using areas, pages, and content elements
    \item Dynamic query processing using \wbpl{} for data-driven content
    \item Modular development approaches using webbaselets
    \item Security and governance considerations for enterprise portal development
    \item Best practices for maintainable and scalable portal architectures
\end{itemize}

The concepts introduced in this part provide the foundation for understanding the Web Spinner engine implementation (Part V) and the development environment provided by Spin the Web Studio (Part VI).

By the end of this part, readers will have a comprehensive understanding of:

\begin{itemize}
\item The complete \wbdl{} language specification and its element hierarchy
\item How \wbpl{} enables dynamic, secure query processing and data binding
\item The modular architecture principles that support enterprise-scale portal development
\item Best practices for organizing and structuring complex portal projects
\item Integration patterns for combining multiple webbaselets into cohesive portals
\end{itemize}

The concepts presented in this part provide the foundation for understanding how the Web Spinner processes \wbdl{} documents and how the Spin the Web Studio enables efficient portal development, topics covered in subsequent parts of this book.
