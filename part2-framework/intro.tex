% Part II: The Framework

\chapter*{Introduction to Part II: The Framework}
\addcontentsline{toc}{chapter}{Introduction to Part II: The Framework}
\label{part:framework}

\begin{quote}
\textit{"The limits of my language mean the limits of my world."} \\
— Ludwig Wittgenstein
\end{quote}

This part opens the hood of the Spin the Web framework, examining its specifications. We will dissect the core components introduced in the architecture overview: the languages that serve as the framework's instruction set, the modular design that enables scalability, and the mechanics of the engine that brings it all to life.

\begin{description}
\item[\textbf{Chapter 5: The WBDL Language}] (\cref{chap:wbdl}) -- Provides a comprehensive introduction to the Webbase Description Language, including its JSON Schema definitions, element hierarchy, and practical usage patterns.

\item[\textbf{Chapter 6: The WBPL Language}] (\cref{chap:wbpl}) -- Explores the Webbase Placeholders Language, which enables dynamic content injection and template-based portal generation.

\item[\textbf{Chapter 7: The WBLL Language}] (\cref{chap:wbll}) -- Defines the Webbase Layout Language, presentation layer, tokens, helpers, and compilation model used to render data into accessible, responsive HTML.

\item[\textbf{Chapter 8: Webbase and Webbaselets}] (\cref{chap:webbaselets}) -- Examines the modular component system that allows for reusable portal elements and cross-platform integration.

\item[\textbf{Chapter 9: The Web Spinner}] (\cref{chap:web-spinner}) -- Details the runtime engine that processes WBDL specifications and generates dynamic web portals, including its architecture, processing pipeline, and performance characteristics.

\item[\textbf{Chapter 10: The Spin the Web Studio}] (\cref{chap:studio}) -- Introduces the integrated development environment for building and managing webbases.

\item[\textbf{Chapter 11: Technology Stack and Implementation}] (\cref{chap:technology}) -- Documents the concrete technology stack (Deno/TypeScript) of the reference Web Spinner, showing how the theoretical mechanics are realized in a working system.
\end{description}

Mastering these specifications is the first step toward building robust, enterprise-grade portals. This section serves as the definitive reference for the framework's specifications, runtime, and tools.
